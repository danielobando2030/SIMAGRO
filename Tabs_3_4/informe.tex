% Options for packages loaded elsewhere
\PassOptionsToPackage{unicode}{hyperref}
\PassOptionsToPackage{hyphens}{url}
\documentclass[
]{article}
\usepackage{xcolor}
\usepackage[margin=1in]{geometry}
\usepackage{amsmath,amssymb}
\setcounter{secnumdepth}{-\maxdimen} % remove section numbering
\usepackage{iftex}
\ifPDFTeX
  \usepackage[T1]{fontenc}
  \usepackage[utf8]{inputenc}
  \usepackage{textcomp} % provide euro and other symbols
\else % if luatex or xetex
  \usepackage{unicode-math} % this also loads fontspec
  \defaultfontfeatures{Scale=MatchLowercase}
  \defaultfontfeatures[\rmfamily]{Ligatures=TeX,Scale=1}
\fi
\usepackage{lmodern}
\ifPDFTeX\else
  % xetex/luatex font selection
\fi
% Use upquote if available, for straight quotes in verbatim environments
\IfFileExists{upquote.sty}{\usepackage{upquote}}{}
\IfFileExists{microtype.sty}{% use microtype if available
  \usepackage[]{microtype}
  \UseMicrotypeSet[protrusion]{basicmath} % disable protrusion for tt fonts
}{}
\makeatletter
\@ifundefined{KOMAClassName}{% if non-KOMA class
  \IfFileExists{parskip.sty}{%
    \usepackage{parskip}
  }{% else
    \setlength{\parindent}{0pt}
    \setlength{\parskip}{6pt plus 2pt minus 1pt}}
}{% if KOMA class
  \KOMAoptions{parskip=half}}
\makeatother
\usepackage{graphicx}
\makeatletter
\newsavebox\pandoc@box
\newcommand*\pandocbounded[1]{% scales image to fit in text height/width
  \sbox\pandoc@box{#1}%
  \Gscale@div\@tempa{\textheight}{\dimexpr\ht\pandoc@box+\dp\pandoc@box\relax}%
  \Gscale@div\@tempb{\linewidth}{\wd\pandoc@box}%
  \ifdim\@tempb\p@<\@tempa\p@\let\@tempa\@tempb\fi% select the smaller of both
  \ifdim\@tempa\p@<\p@\scalebox{\@tempa}{\usebox\pandoc@box}%
  \else\usebox{\pandoc@box}%
  \fi%
}
% Set default figure placement to htbp
\def\fps@figure{htbp}
\makeatother
\setlength{\emergencystretch}{3em} % prevent overfull lines
\providecommand{\tightlist}{%
  \setlength{\itemsep}{0pt}\setlength{\parskip}{0pt}}
\usepackage{float}
\usepackage{arydshln}
\usepackage{tabu}
\usepackage{xcolor}
\usepackage{fontspec}
\usepackage{booktabs}
\usepackage{fancyhdr}
\usepackage{graphicx}
\definecolor{mygreen}{RGB}{26,73,34}
\definecolor{gray}{RGB}{128,128,128}
\definecolor{green2}{RGB}{13,141,56}
\pagestyle{fancy}
\fancyfoot{}
\usepackage{colortbl}
\usepackage{adjustbox}
\setlength{\headheight}{2cm}
\fancyhead[C]{\includegraphics[width=\textwidth]{www/logo_3.png}}
\renewcommand{\headrule}{\color{mygreen}\hrule width\headwidth height\headrulewidth}
\usepackage{eso-pic}
\usepackage[utf8]{inputenc}
\usepackage{booktabs}
\usepackage{longtable}
\usepackage{array}
\usepackage{multirow}
\usepackage{wrapfig}
\usepackage{float}
\usepackage{colortbl}
\usepackage{pdflscape}
\usepackage{tabu}
\usepackage{threeparttable}
\usepackage{threeparttablex}
\usepackage[normalem]{ulem}
\usepackage{makecell}
\usepackage{xcolor}
\usepackage{bookmark}
\IfFileExists{xurl.sty}{\usepackage{xurl}}{} % add URL line breaks if available
\urlstyle{same}
\hypersetup{
  hidelinks,
  pdfcreator={LaTeX via pandoc}}

\author{}
\date{\vspace{-2.5em}}

\begin{document}

\begin{flushright}
\textcolor{gray}{Informe generado el: \textbf{08 de noviembre de 2025}}
\end{flushright}

\fontsize{14}{14}\selectfont\textcolor{mygreen}{Análisis de precios de alimentos por ciudad}\\

\fontsize{12}{12}\selectfont\textcolor{green2}{Comparación de precios de alimentos entre ciudades respecto a Bogotá}\\

\begin{center}\rule{0.5\linewidth}{0.5pt}\end{center}

\subsubsection{Parámetros de
generación}\label{paruxe1metros-de-generaciuxf3n}

\begin{table}[H]
\centering
\begin{tabular}{ll}
\toprule
Parámetro & Valor\\
\midrule
Producto & NA\\
Año & NA\\
\bottomrule
\end{tabular}
\end{table}

\begin{center}\rule{0.5\linewidth}{0.5pt}\end{center}

\begin{verbatim}
## NA
\end{verbatim}

\begin{center}\rule{0.5\linewidth}{0.5pt}\end{center}

\subsubsection{Visualización
comparativa}\label{visualizaciuxf3n-comparativa}

\begin{verbatim}
## ⚠️ Error al renderizar el gráfico:  $ operator is invalid for atomic vectors
\end{verbatim}

\begin{center}\rule{0.5\linewidth}{0.5pt}\end{center}

\subsubsection{Tabla de resultados
principales}\label{tabla-de-resultados-principales}

\begin{verbatim}
## ⚠️ Error al generar la tabla:  missing value where TRUE/FALSE needed
\end{verbatim}

\begin{center}\rule{0.5\linewidth}{0.5pt}\end{center}

\subsubsection{Interpretación}\label{interpretaciuxf3n}

\fontsize{10}{10}\selectfont

\begin{itemize}
\tightlist
\item
  Valores positivos indican precios \textbf{mayores que Bogotá}.\\
\item
  Valores negativos reflejan precios \textbf{menores que Bogotá}.\\
\item
  El tamaño del círculo representa la \textbf{variabilidad interna
  (desviación estándar)}.\\
\item
  Las ciudades con valores extremos permiten identificar diferencias
  relevantes entre mercados.
\end{itemize}

\begin{center}\rule{0.5\linewidth}{0.5pt}\end{center}

\fontsize{8}{8}\selectfont\textcolor{mygreen}{Fuente: Cálculos propios con base en el Sistema de Información de Precios y Abastecimiento del Sector Agropecuario (SIPSA) – DANE.}\\

\fontsize{8}{8}\selectfont\textcolor{mygreen}{Bogotá se toma como referencia por ser el principal mercado mayorista del país.}\\

\fontsize{8}{8}\selectfont\textcolor{mygreen}{Informe generado automáticamente mediante la aplicación Shiny – FAO 2025.}\\

\AddToShipoutPictureBG*{\includegraphics[width=\paperwidth,height=3cm,keepaspectratio]{www/logo_2.png}}

\end{document}
