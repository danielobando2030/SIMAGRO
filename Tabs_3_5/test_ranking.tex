% Options for packages loaded elsewhere
\PassOptionsToPackage{unicode}{hyperref}
\PassOptionsToPackage{hyphens}{url}
\documentclass[
]{article}
\usepackage{xcolor}
\usepackage[margin=1in]{geometry}
\usepackage{amsmath,amssymb}
\setcounter{secnumdepth}{-\maxdimen} % remove section numbering
\usepackage{iftex}
\ifPDFTeX
  \usepackage[T1]{fontenc}
  \usepackage[utf8]{inputenc}
  \usepackage{textcomp} % provide euro and other symbols
\else % if luatex or xetex
  \usepackage{unicode-math} % this also loads fontspec
  \defaultfontfeatures{Scale=MatchLowercase}
  \defaultfontfeatures[\rmfamily]{Ligatures=TeX,Scale=1}
\fi
\usepackage{lmodern}
\ifPDFTeX\else
  % xetex/luatex font selection
\fi
% Use upquote if available, for straight quotes in verbatim environments
\IfFileExists{upquote.sty}{\usepackage{upquote}}{}
\IfFileExists{microtype.sty}{% use microtype if available
  \usepackage[]{microtype}
  \UseMicrotypeSet[protrusion]{basicmath} % disable protrusion for tt fonts
}{}
\makeatletter
\@ifundefined{KOMAClassName}{% if non-KOMA class
  \IfFileExists{parskip.sty}{%
    \usepackage{parskip}
  }{% else
    \setlength{\parindent}{0pt}
    \setlength{\parskip}{6pt plus 2pt minus 1pt}}
}{% if KOMA class
  \KOMAoptions{parskip=half}}
\makeatother
\usepackage{graphicx}
\makeatletter
\newsavebox\pandoc@box
\newcommand*\pandocbounded[1]{% scales image to fit in text height/width
  \sbox\pandoc@box{#1}%
  \Gscale@div\@tempa{\textheight}{\dimexpr\ht\pandoc@box+\dp\pandoc@box\relax}%
  \Gscale@div\@tempb{\linewidth}{\wd\pandoc@box}%
  \ifdim\@tempb\p@<\@tempa\p@\let\@tempa\@tempb\fi% select the smaller of both
  \ifdim\@tempa\p@<\p@\scalebox{\@tempa}{\usebox\pandoc@box}%
  \else\usebox{\pandoc@box}%
  \fi%
}
% Set default figure placement to htbp
\def\fps@figure{htbp}
\makeatother
\setlength{\emergencystretch}{3em} % prevent overfull lines
\providecommand{\tightlist}{%
  \setlength{\itemsep}{0pt}\setlength{\parskip}{0pt}}
\usepackage[utf8]{inputenc}
\usepackage[T1]{fontenc}
\usepackage[spanish,es-tabla]{babel}
\usepackage{lmodern}
\usepackage{booktabs}
\usepackage[table]{xcolor}
\usepackage{colortbl}
\usepackage{float}
\selectlanguage{spanish}
\usepackage{bookmark}
\IfFileExists{xurl.sty}{\usepackage{xurl}}{} % add URL line breaks if available
\urlstyle{same}
\hypersetup{
  pdftitle={Ranking mensual de precios mayoristas por ciudad},
  hidelinks,
  pdfcreator={LaTeX via pandoc}}

\title{Ranking mensual de precios mayoristas por ciudad}
\author{}
\date{\vspace{-2.5em}}

\begin{document}
\maketitle

\begin{center}\rule{0.5\linewidth}{0.5pt}\end{center}

\section{Ranking mensual de precios mayoristas por
ciudad}\label{ranking-mensual-de-precios-mayoristas-por-ciudad}

\section{\texorpdfstring{\textbf{Habichuela ---
2025}}{Habichuela --- 2025}}\label{habichuela-2025}

\textcolor{#7B1FA2}{\large Ranking mensual de precios mayoristas por ciudad}

Este informe presenta la posición relativa de los precios mayoristas de
cada ciudad respecto a las demás durante el año seleccionado. Bogotá se
destaca en color púrpura.

\begin{center}\rule{0.5\linewidth}{0.5pt}\end{center}

\section{Visualización general}\label{visualizaciuxf3n-general}

\begin{verbatim}
## No hay gráfico disponible para los parámetros seleccionados.
\end{verbatim}

\begin{center}\rule{0.5\linewidth}{0.5pt}\end{center}

\section{Tabla de ranking por mes}\label{tabla-de-ranking-por-mes}

\begin{table}[!h]
\centering
\caption{\label{tab:tabla}Posiciones de las ciudades según el precio promedio mensual}
\centering
\begin{tabular}[t]{lccc}
\toprule
Mes & Ciudad & Precio promedio & Posición\\
\midrule
\cellcolor{gray!10}{2025-01} & \cellcolor{gray!10}{Bogotá} & \cellcolor{gray!10}{\$3.200} & \cellcolor{gray!10}{1}\\
2025-02 & Cali & \$2.900 & 2\\
\bottomrule
\end{tabular}
\end{table}

\begin{center}\rule{0.5\linewidth}{0.5pt}\end{center}

\section{Interpretación}\label{interpretaciuxf3n}

\begin{itemize}
\tightlist
\item
  Las posiciones más bajas (1, 2, 3) representan los \textbf{precios más
  altos} del mercado.\\
\item
  Las posiciones más altas (mayores valores de ranking) indican
  \textbf{precios más bajos}.\\
\item
  Bogotá se visualiza con borde dorado para resaltar su comportamiento
  frente a otras ciudades.\\
\item
  Este ranking permite identificar diferencias espaciales en los precios
  mayoristas de cada producto.
\end{itemize}

\begin{center}\rule{0.5\linewidth}{0.5pt}\end{center}

\section{Fuente y notas
metodológicas}\label{fuente-y-notas-metodoluxf3gicas}

\small

Fuente: \textbf{Sistema de Información de Precios y Abastecimiento del
Sector Agropecuario (SIPSA)} -- DANE.\\
Elaboración propia. Los precios corresponden al valor mayorista por
kilogramo de producto de primera calidad registrado en cada central de
abastos.\\
\newline \newline \emph{Informe generado automáticamente con la
aplicación Shiny -- FAO 2025.}

\end{document}
